\section{The widget Viewer}

\cbstart
This widget is part of the {\em Imaging} Add-On and it displays an image.

\subsection{Construction}

The primitive {\tt Viewer} is used to build a new Viewer widget. Its syntax is:\\

{\tt Viewer} {\tt image | list}

The only input can be either an {\it Image} object or a list. The list specify the size in pixel of the widget. If an image is given, the widget will adapt it
size to fit the full image.

\subsection{Methods}

The Viewer widget has two specific methods:\\

{\bf display\index{Viewer@\textbf{Viewer}!Methods!display}} 
\begin{verbatimtab}
$viewer~display image (word)
\end{verbatimtab}
Display the image given as first input. If a second input is specified, it must be one of the following words : {\tt "adapt "center "scale "scroll}. This second input give the way the widget shall display the image. By default it is {\tt "adapt}. The following table explain the different style of display:\\

\begin{table}[h!]
\centering
\begin{tabular}{|c|p{6cm}|}
\hline
\bf Style & \bf Purpose \\
\hline
\tt "adapt\index{Viewer@\textbf{Viewer}!Style!adapt} &  Adapt the size of the widget to fit the complete image. Resize the window to fit.\\
\hline
\tt "center\index{Viewer@\textbf{Viewer}!Style!center} & Display the image without resizing the widget. The image is centered on the center of the widget.\\
\hline
\tt "scale\index{Viewer@\textbf{Viewer}!Style!scale} &  Display the image scaled to fit in the widget.\\
\hline
\tt "scroll\index{Viewer@\textbf{Viewer}!Style!scroll} &  Keep the current widget size, display scrollbars for the user to see the image.\\
\hline
\end{tabular}
\caption{Viewer's display styles}
\end{table}

{\bf resize.to\index{Viewer@\textbf{Viewer}!Methods!resize.to}} 
\begin{verbatimtab}
$viewer~resize.to list
\end{verbatimtab}
Resize the widget to a given size. The method has for effect to ask the window
where the widget is to resize as well.\\

\subsection{Configuration}

This widget don't have any specific configuration.

\subsection{Hooks}

This widget don't have specific hooks.

\subsection{Example}

\example{ex50}	
\begin{listing}{1}
use 'List Processing'

if (llength :Args) = 2 { 

	use 'GUI' 'Imaging'

	make "my.image Image lindex :Args 2 
	if (is.image :my.image) { 
        make "win Window "titled (lindex :Args 2) [100 100] "not.resizable
		make "the.viewer Viewer :my.image 
		Glue :win "top [] :the.viewer 
		$win~show    
  	} { 
		print 'Image file not reconized ... maybe not an image' 
	} 
} { 
	print 'USAGE : image' 
}
\end{listing}

\begin{figure}[h!]
\centering
\psfig{figure=images/eps/ex3-50.eps,width=110pt,height=68pt}
\label{s50}
\caption{The Viewer widget}
\end{figure}


\cbend
