\section{The widget Frame}

A "Frame" is a container widget displaying a relief border. This border could be raised, lowered or flattened.

\subsection{Construction}

The primitive {\tt Frame} is used to build a new Frame widget. Its syntax is :\\

{\tt Entry} {\it word (list (words)) }\\

The first input is the relief style of the frame, and it must be a valid word : {\tt "flattened "raised "bordered} or {\tt "lowered}. The second input, if specified, is the size of the widget, consisting of a list of two elements (width height).  Any other inputs make up the flags.

\subsection{Methods}

A Frame widget has three methods :\\

{\bf reglue\index{Frame@\textbf{Frame}!Methods!reglue}} 
\begin{verbatimtab}
$frame~reglue
\end{verbatimtab}
This primitive asks the {\em geometry manager} to glue all the widgets within the Frame a second time. This primitive is useful when a child of the Frame has been removed and new gluing
is required.\\

{\bf relief\index{Frame@\textbf{Frame}!Methods!relief}} 
\begin{verbatimtab}
$frame~relief word (word)
\end{verbatimtab}
Set or get (according to the value of the first input : {\tt "set} or {\tt "get}) the relief of the border. When setting a value, the second input must be one of the valid words : {\tt "lowered "flattened "bordered} or {\tt "raised}.\\

{\bf widgets\index{Frame@\textbf{Frame}!Methods!widgets}} 
\begin{verbatimtab}
$frame~wigets
\end{verbatimtab}
Output all the widgets glued on the Frame.\\

\subsection{Configuration}

\begin{table}[ht]
\centering
\begin{tabular}{|c|p{5cm}|p{5cm}|}
\hline
\bf Configuration & \bf Purpose & \bf Value \\
\hline
\tt "level\index{Frame@\textbf{Frame}!Configuration!level} & Set or get the level of the relief & a number\\
\hline
\end{tabular}
\caption{Frame's configuration}
\end{table}

\subsection{Hooks}

This widget doesn't have any hooks.

\subsection{Example}

\example{ex36}	
\begin{listing}{1}
make "win Window "titled 'Frame' [100 100]
make "f Frame "bordered [70 70]
$f~config "level "set 5
Glue :win "top [] :f
$win~show
\end{listing}

On line 3, we set the level of relief of the border to 5.

\begin{figure}[h!]
\centering
\psfig{figure=images/eps/ex3-18.eps}
\label{s36}
\caption{A Frame widget with a bordered relief}
\end{figure}

\example{ex37}	
\begin{listing}{1}
make "win Window "titled 'Frame' [100 100]
make "f Frame "raised [70 70]
$f~config "level "set 2
Glue :win "top [] :f
$win~show
\end{listing}

On line 3, we set the level of the relief of the border to 5.

\begin{figure}[h!]
\centering
\psfig{figure=images/eps/ex3-19.eps}
\label{s37}
\caption{A Frame widget with a raised relief}
\end{figure}

\example{ex38}	
\begin{listing}{1}
make "win Window "titled 'Frame' [100 100]
make "f Frame "lowered [70 70]
Glue :win "top [] :f
$win~show
\end{listing}

A lower relief gives a nice sunken feature to the frame.

\begin{figure}[h!]
\centering
\psfig{figure=images/eps/ex3-20.eps}
\label{s38}
\caption{A Frame widget with a lowered relief}
\end{figure}

\example{ex39}	
\begin{listing}{1}
make "win Window "titled 'Frame' [100 100]
make "f Frame "flattened [70 70]
Glue :win "top [] :f
$win~show
\end{listing}

\begin{figure}[h!]
\centering
\psfig{figure=images/eps/ex3-21.eps}
\label{s39}
\caption{A Frame widget without a border}
\end{figure}
