
\section{The widget Entry}

This widget is a simple labeled text field. After modifying the text by pressing the {\em Enter} key or by changing the focus, an event will be generated and the linked variable will be updated.

\subsection{Construction}

The primitive {\tt Entry} is used to build a new Entry widget. Its syntax is :\\

{\tt Entry} {\it string \verb?|? word word (list (words)) }\\

The first input is the label of the widget. The second input is the name of the linked variable. If specified, a fourth input will be a list containing the size of the label in characters and the size of the entry field in characters.  All other inputs would make up the flags.\\

The widget adapt itself to the value of the linked variable. If the value is a number, only number will be allowed in the Entry and once changed by the user, the new value set to the linked variable will be a number.

\subsection{Methods}

{\bf invoke\index{Entry@\textbf{Entry}!Methods!invoke}} 
\begin{verbatimtab}
$entry~invoke
\end{verbatimtab}
Invoke the widget as though the text was modified by the user.\\

{\bf entry\index{Entry@\textbf{Entry}!Methods!entry}} 
\begin{verbatimtab}
$entry~entry word  word (string | word)
\end{verbatimtab}
Set or get the configuration of the entry field. The first input must be the word {\tt "align} or {\tt "expand}. When {\tt "align} is used, the method will get or set the alignment of the label within the widget. When the first input is the word {\tt "expand}, the expand property of the field will be set or get, and a boolean value will be required.  When setting the value, the third input must be the word : {\tt "left "right} or {\tt "center}.\\

{\bf label\index{Entry@\textbf{Entry}!Methods!label}} 
\begin{verbatimtab}
$entry~label word word (string | word)
\end{verbatimtab}
Set or get the configuration of the label. The first input must be the word {\tt "align} or {\tt "text}. When {\tt "align} is used, the method will get or set the alignment of the entry field within the widget. If the first input is the word {\tt "text}, the label string will be set or get. The third input when setting the value must be the word : {\tt "left "right} or {\tt "center} when working on the alignment, otherwise it must be a string.

\subsection{Configuration}

\begin{table}[ht]
\centering
\begin{tabular}{|c|p{5cm}|p{5cm}|}
\hline
\bf Configuration & \bf Purpose & \bf Value \\
\hline
\tt "value\index{Entry@\textbf{Entry}!Configuration!value} & Set or get the value of the entry field & a string or word\\
\hline
\tt "variable\index{Entry@\textbf{Entry}!Configuration!variable} & Set or get the linked variable of the widget & a word\\
\hline
\end{tabular}
\caption{Entry's configuration}
\end{table}

\newpage

\subsection{Hooks}

\begin{table}[h!]
\centering
\begin{tabular}{|l|p{3cm}|p{8.5cm}|}
\hline
\bf Name & \bf Description & \bf Function prototype \\
\hline
\tt "changed\index{Entry@\textbf{Entry}!Hooks!changed} & The string in the entry field have been modified & {\tt
to changed :src :old :new\newline
; src is the widget object\newline
; old is the old string\newline
; new is the new string entered by the user\newline
end
}\\
\hline
\end{tabular}
\caption{Entry's hooks}
\end{table}

\subsection{Example}

\example{ex34}	
\begin{listing}{1}
make "win Window "titled 'Entry' [100 100]
make "f Frame
make "name Entry 'What\'s your name ?' "friend
Hook :name "changed {
	Info "none ["Hello] '' 'Hello' :friend '!'	
}
Glue :f "top [] :name
Glue :win "top [] :f
$win~show
\end{listing}

When the user hits the {\em Enter} key after an update in the entry field, the hook {\tt changed} will be called.

\begin{figure}[h!]
\centering
\psfig{figure=images/eps/ex3-16.eps}
\label{s34}
\caption{A Entry widget}
\end{figure}

The next example play with the linked variable :

\example{ex35}	
\begin{listing}{1}
to IncrTime :src :s :t
	if :s {
		make "Time string (parse.number :Time) + :t	
	} {
		make "Time string (parse.number :Time) - :t				
	}
end

make "win Window "titled 'Entry' [100 100]
make "box Box 'Test to do'
$box~config "expand.x "set true
make "c1 CheckBox 'Computer' "computer
make "c2 CheckBox 'AC' "ac
make "c3 CheckBox 'Power' "power
Hook :c1 "invoked "IncrTime 10
Hook :c2 "invoked "IncrTime 20
Hook :c3 "invoked "IncrTime 30
Glue :box "top [] :c1 :c2 :c3
make "f Frame
make "Time '0'
make "name Entry 'Time :' "Time [0 4]
Glue :f "top [] :name
Glue :win "top [] :box :f
$win~show
\end{listing}

The function {\tt IncrTime} is called each time one of the CheckBoxes is checked or unchecked. This function modifies the variable which is always a string.

\begin{figure}[h!]
\centering
\psfig{figure=images/eps/ex3-17.eps}
\label{s35}
\caption{A Entry widget updated by its linked variable}
\end{figure}
