\chapter{Supports}

This Add-on give access to several other primitives as well as some new objects used by the widgets.

\section{Fonts}

Fonts management under \squirrel is done by using a set of primitive and a new object: Font.

\subsection{Primitives}

{\bf Font.init\index{Fonts@\textbf{Fonts}!Font.init}} 
\begin{verbatimtab}
Font.init
\end{verbatimtab}
This primitive in mandatory in \squirrel to use with fonts.  When fonts are used, the primitive should always be one of the first things a script performs.\\

{\bf Font.families\index{Fonts@\textbf{Fonts}!Font.families}} 
\begin{verbatimtab}
Font.families
\end{verbatimtab}
Output a list of all the font families installed on the computer.\\

\cbstart
{\bf Font.exists\index{Fonts@\textbf{Fonts}!Font.exists}} 
\begin{verbatimtab}
Font.exists string
\end{verbatimtab}
Output {\tt true} if a font family given as input to the primitive is installed on the system.\\
\cbend

{\bf Font.styles\index{Fonts@\textbf{Fonts}!Font.styles}} 
\begin{verbatimtab}
Font.styles string
\end{verbatimtab}
Output a list of all the styles available to a font family given as input to the primitive.\\

\subsection{Font object}

The primitive {\tt Font} creates a new font object for a specified font family. During its lifetime, a font object can change family. The syntax of this primitive is : \\

{\tt Font} {\it string} \\

A Font object has several methods:\\

{\bf aliasing\index{Font@\textbf{Font}!Methods!aliasing}} 
\begin{verbatimtab}
$font~aliasing word (word)
\end{verbatimtab}
Get or set whether the font is using anti-aliasing or not. The first input must be the word {\tt "get} or {\tt "set}.  The second input must be the word {\tt "on} or {\tt "off}.\\

{\bf direction\index{Font@\textbf{Font}!Methods!direction}} 
\begin{verbatimtab}
$font~direction word (word)
\end{verbatimtab}
Get or set if the font has direction. The first input must be the word {\tt "get} or {\tt "set}.  The second input must be the word {\tt "left2right} or {\tt "right2left}.\\
 
{\bf encoding\index{Font@\textbf{Font}!Methods!encoding}} 
\begin{verbatimtab}
$font~encoding word (number)
\end{verbatimtab}
Get or set if the font is encoding. The first input must be the word {\tt "get} or {\tt "set}.  The second input must be a number from the following table :

\newpage

\begin{table}[h]
\centering
\begin{tabular}{|c|c|}
\hline
UNICODE\_UTF8 & 0\\
\hline
ISO\_8859\_1 & 1\\
\hline
ISO\_8859\_2 & 2\\
\hline
ISO\_8859\_3 & 3\\
\hline
ISO\_8859\_4 & 4\\
\hline
ISO\_8859\_5 & 5\\
\hline
ISO\_8859\_6 & 6\\
\hline
ISO\_8859\_7 & 7\\
\hline
ISO\_8859\_8 & 8\\
\hline
ISO\_8859\_9 & 9\\
\hline
ISO\_8859\_10 & 10\\
\hline
MACINTOSH\_ROMAN & 11\\
\hline
\end{tabular}
\caption{Font encoding}
\end{table} 
 
{\bf family\index{Font@\textbf{Font}!Methods!family}} 
\begin{verbatimtab}
$font~family word (string)
\end{verbatimtab}
Get or set if it's the family of the font object. The first input must be the word {\tt "get} or {\tt "set}. The second input (if any) is the name of the family.\\

{\bf rotation\index{Font@\textbf{Font}!Methods!rotation}} 
\begin{verbatimtab}
$font~rotation word (number)
\end{verbatimtab}
Get or set if there's rotation to the font object. The first input must be the word {\tt "get} or {\tt "set}. The second input (if any) is a number between 0 and 360.\\

{\bf shear\index{Font@\textbf{Font}!Methods!shear}} 
\begin{verbatimtab}
$font~shear word (number)
\end{verbatimtab}
Get or set if there's shear to the font object. The first input must be the word {\tt "get} or {\tt "set}. The second input (if any) is a number between 45 and 135.\\

{\bf size\index{Font@\textbf{Font}!Methods!size}} 
\begin{verbatimtab}
$font~size word (number)
\end{verbatimtab}
Get or set if there's size to the font object. The first input must be the word {\tt "get} or {\tt "set}. The second input (if any) is a number.\\

{\bf spacing\index{Font@\textbf{Font}!Methods!spacing}} 
\begin{verbatimtab}
$font~spacing word (word)
\end{verbatimtab}
Get or set if there's spacing to the font object. The first input must be the word {\tt "get} or {\tt "set}. The second input (if any) must be one of the valid word : {\tt "char "string "bitmap "fixed}.\\ 

{\bf style\index{Font@\textbf{Font}!Methods!style}} 
\begin{verbatimtab}
$font~style word (word)
\end{verbatimtab}
Get or set if there's style to the font object. The first input must be the word {\tt "get} or {\tt "set}. Style varies from font but it's usually the words or strings : {\tt "Regular "Roman "Bold 'Bold Italic'} or {\tt "Italic}. \\

\subsection{A Little example}

The next example shows how simple it is to build a Font browser. The window is composed of two SimpleLists, one for the families and one for the style. A Text widget, whose font is changed each time the user selects a font or a style, displays a string :

\example{ex46}	
\begin{listing}{1}
Font.init

make "win Window "titled 'Fonts' [100 100]
make "frame Frame
make "Families Font.families
make "Family lindex :Families 1 

make "theFont Font :Family
$theFont~style "set lindex (Font.styles :Family) 1
$theFont~size "set 15

make "fbox Box 'Family'
make "families SimpleList "single "right "Family :Families [0 6]
Hook :families "selected {
	$styles~items "set (Font.styles :Family)
	make "Style lindex (Font.styles :Family) 1
}
Glue :fbox "top [] :families
make "sbox Box 'Styles'
make "Style "Roman
make "styles SimpleList "single "right "Style (Font.styles :Family) [0 4]
Hook :styles "selected {
	$theFont~family "set :Family
	$theFont~style "set :Style		
	$label~config "font "set :theFont
}
make "label Text 'Select a family and a style!'
$label~config "font "set :theFont
$label~config "bgcolor "set ($frame~config "bgcolor "get)
$label~config "expand.x "set true
$label~justify "set "center
Glue :sbox "top [] :styles
Glue :frame "left [] :fbox :sbox
Glue :win "top [] :frame :label
$win~show
\end{listing}

\begin{figure}[h!]
\centering
\psfig{figure=images/eps/ex3-28.eps}
\label{s46}
\caption{Browsing the installed font}
\end{figure}

\section{Color List}

The file {\tt /boot/apps/Squirrel/Libraries/Colors.sqi} list several useful colors. You may load this file at the beginning of your script in order to use them. You will find
several standard colors like red, black etc ... as well as the standard colors of the
\beos interface.\\

You will notice by looking at this file, that each color is stored in a variable :\\

{\tt make "LightBlue [64 162 255 255]}\\

A color is a well known mix of the three colors : red , blue and green. In the list, these are always the first three elements. A fourth element is not mandatory.  If specified, however, it's the
value of the Alpha channel. 

\section{Primitives}

Several primitives are added to \squirrel by the GUI Add-on to test objects for membership to certain types or set the focus :\\

\cbstart
{\bf Busy\index{Add-On@\textbf{Add-On}!Primitives!Busy}} 
\begin{verbatimtab}
Busy boolean (word)
\end{verbatimtab}
Set or unset the application cursor to an animated {\it busy} cursor. The first input indicate by {\tt true} or {\tt false} if the cursor shall be animated or not. If a second input is given, it must be {\tt "spinwheel} or {\tt "watch}, it indicates the type if cursor to use. By default, {\tt "spinwheel} is used.\\
\cbend

{\bf Focus\index{Add-On@\textbf{Add-On}!Primitives!Focus}} 
\begin{verbatimtab}
Focus widget
\end{verbatimtab}
Set the keyboard focus on a widget.\\

{\bf is.banner\index{Add-On@\textbf{Add-On}!Primitives!is.banner}} 
\begin{verbatimtab}
is.banner thing
\end{verbatimtab}
Output {\tt true} if the input is a Banner widget, otherwise output {\tt false}.\\

{\bf is.barberpole\index{Add-On@\textbf{Add-On}!Primitives!is.barberpole}} 
\begin{verbatimtab}
is.barberpole thing
\end{verbatimtab}
Output {\tt true} if the input is a BarberPole widget, otherwise output {\tt false}.\\

{\bf is.box\index{Add-On@\textbf{Add-On}!Primitives!is.box}} 
\begin{verbatimtab}
is.box thing
\end{verbatimtab}
Output {\tt true} if the input is a Box widget, otherwise output {\tt false}.\\

{\bf is.button\index{Add-On@\textbf{Add-On}!Primitives!is.button}} 
\begin{verbatimtab}
is.button thing
\end{verbatimtab}
Output {\tt true} if the input is a Button widget, otherwise output {\tt false}.\\

{\bf is.checkbox\index{Add-On@\textbf{Add-On}!Primitives!is.checkbox}} 
\begin{verbatimtab}
is.checkbox thing
\end{verbatimtab}
Output {\tt true} if the input is a CheckBox widget, otherwise output {\tt false}.\\

{\bf is.colorcontrol\index{Add-On@\textbf{Add-On}!Primitives!is.colorcontrol}} 
\begin{verbatimtab}
is.colorcontrol thing
\end{verbatimtab}
Output {\tt true} if the input is a ColorControl widget, otherwise output {\tt false}.\\

{\bf is.container\index{Add-On@\textbf{Add-On}!Primitives!is.container}} 
\begin{verbatimtab}
is.container thing
\end{verbatimtab}
Output {\tt true} if the input is a container widget, otherwise output {\tt false}.\\

{\bf is.droplist\index{Add-On@\textbf{Add-On}!Primitives!is.droplist}} 
\begin{verbatimtab}
is.droplist thing
\end{verbatimtab}
Output {\tt true} if the input is a DropList widget, otherwise output {\tt false}.\\

{\bf is.entry\index{Add-On@\textbf{Add-On}!Primitives!is.entry}} 
\begin{verbatimtab}
is.entry thing
\end{verbatimtab}
Output {\tt true} if the input is an Entry widget, otherwise output {\tt false}.\\

{\bf is.font\index{Add-On@\textbf{Add-On}!Primitives!is.font}} 
\begin{verbatimtab}
is.font thing
\end{verbatimtab}
Output {\tt true} if the input is a Font object, otherwise output {\tt false}.\\

{\bf is.frame\index{Add-On@\textbf{Add-On}!Primitives!is.frame}} 
\begin{verbatimtab}
is.frame thing
\end{verbatimtab}
Output {\tt true} if the input is a Frame widget, otherwise output {\tt false}.\\

{\bf is.font\index{Add-On@\textbf{Add-On}!Primitives!is.font}} 
\begin{verbatimtab}
is.font thing
\end{verbatimtab}
Output {\tt true} if the input is a Font object, otherwise output {\tt false}.\\

{\bf is.memo\index{Add-On@\textbf{Add-On}!Primitives!is.memo}} 
\begin{verbatimtab}
is.memo thing
\end{verbatimtab}
Output {\tt true} if the input is a Memo widget, otherwise output {\tt false}.\\

{\bf is.menu\index{Add-On@\textbf{Add-On}!Primitives!is.menu}} 
\begin{verbatimtab}
is.menubar thing
\end{verbatimtab}
Output {\tt true} if the input is a Menu widget, otherwise output {\tt false}.\\

{\bf is.menubar\index{Add-On@\textbf{Add-On}!Primitives!is.menubar}} 
\begin{verbatimtab}
is.menubar thing
\end{verbatimtab}
Output {\tt true} if the input is a Menubar widget, otherwise output {\tt false}.\\

{\bf is.menubar\index{Add-On@\textbf{Add-On}!Primitives!is.menubar}} 
\begin{verbatimtab}
is.menubar thing
\end{verbatimtab}
Output {\tt true} if the input is a Menubar widget, otherwise output {\tt false}.\\

\cbstart
{\bf is.odometer\index{Odometer@\textbf{Odometer}!Primitives!is.odometer}} 
\begin{verbatimtab}
is.odometer thing
\end{verbatimtab}
Output {\tt true} if the input is an Odometer widget, otherwise output {\tt false}.\\
\cbend

{\bf is.radiobutton\index{Add-On@\textbf{Add-On}!Primitives!is.radiobutton}} 
\begin{verbatimtab}
is.radiobutton thing
\end{verbatimtab}
Output {\tt true} if the input is a Radiobutton widget, otherwise output {\tt false}.\\

{\bf is.statusbar\index{Add-On@\textbf{Add-On}!Primitives!is.statusbar}} 
\begin{verbatimtab}
is.statusbar thing
\end{verbatimtab}
Output {\tt true} if the input is a StatusBar widget, otherwise output {\tt false}.\\

{\bf is.text\index{Add-On@\textbf{Add-On}!Primitives!is.text}} 
\begin{verbatimtab}
is.text thing
\end{verbatimtab}
Output {\tt true} if the input is a Text widget, otherwise output {\tt false}.\\

\cbstart
{\bf is.viewer\index{Viewer@\textbf{Viewer}!Primitives!is.viewer}} 
\begin{verbatimtab}
is.viewer thing
\end{verbatimtab}
Output {\tt true} if the input is a Viewer widget, otherwise output {\tt false}.\\
\cbend

{\bf is.widget\index{Add-On@\textbf{Add-On}!Primitives!is.widget}} 
\begin{verbatimtab}
is.widget thing
\end{verbatimtab}
Output {\tt true} if the input is a widget, otherwise output {\tt false}.\\

{\bf is.window\index{Add-On@\textbf{Add-On}!Primitives!is.window}} 
\begin{verbatimtab}
is.window thing
\end{verbatimtab}
Output {\tt true} if the input is a Menubar widget, otherwise output {\tt false}.
