
\section{The widget Box}

A "Box" is a container widget which draws a labeled border around its children. A Box has three styles of border ({\tt "plain "fancy} or {\tt "none}). The label drawn by the widget is usually text, but it could also be another widget.

\subsection{Construction}

The primitive {\tt Box} is used to build a new Box widget. Its syntax is :\\

{\tt Box} {\it word \verb?|? widget (list (words))}\\

The first input is the label of the Box; it could be either a string or a widget. The primitive accepts a second and third input if needed. They are the size of the widget in a list in pixels (width and height) and a set of the usual widget flags. 

\subsection{Methods}

A Box widget has three methods :\\

{\bf reglue\index{Box@\textbf{Box}!Methods!reglue}} 
\begin{verbatimtab}
$box~reglue
\end{verbatimtab}
This primitive asks the {\em geometry manager} to glue all the widgets within the Box a second time. This primitive is useful when a child from the Box is removed and new gluing is needed.\\

{\bf style\index{Box@\textbf{Box}!Methods!style}} 
\begin{verbatimtab}
$box~style word (word)
\end{verbatimtab}
Set or get (according to the value of the first input : {\tt "set} or {\tt "get}) the style of the border. When setting a value, the second input must be one of the valid words : {\tt "plain "fancy} or {\tt "none}.\\

{\bf widgets\index{Box@\textbf{Box}!Methods!widgets}} 
\begin{verbatimtab}
$box~wigets
\end{verbatimtab}
Output all the widgets glued on the Box.\\

\subsection{Configuration}

Only one configuration is added to the Box widget :
\begin{table}[ht]
\centering
\begin{tabular}{|c|p{5cm}|p{5cm}|}
\hline
\bf Configuration & \bf Purpose & \bf Value \\
\hline
\tt "label\index{Box@\textbf{Box}!Configuration!label} & Set or get the label of the widget & could be a word, a string or another widget.\\
\hline
\end{tabular}
\caption{Box's configuration}
\end{table}

\subsection{Hooks}

This widget has nothing particular.

\subsection{Example}

\example{ex23}	
\begin{listing}{1}
make "win Window "titled 'Box' [100 100]
make "box Box 'A Box'
make "frame Frame "flattened [50 50]
Glue :box "top [] :frame
Glue :win "top [] :box
$win~show
\end{listing}

On line 3, we create a 50x50 pixel Frame widget.  It's used in this example to fill the Box widget.

\begin{figure}[h!]
\centering
\psfig{figure=images/eps/ex3-3.eps}
\label{s23}
\caption{Box with a text label}
\end{figure}

In the next example, we use a button to label the Box :

\example{ex24}	
\begin{listing}{1}
make "win Window "titled 'Banner' [100 100]
make "label Button 'Click me'
make "box Box :label
make "frame Frame "flattened [50 50]
Glue :box "top [] :frame
Glue :win "top [] :box
$win~show
\end{listing}

The primitive {\tt Box} on line 3 calls the button object instead of a simple string for labeling the Box.

\begin{figure}[h!]
\centering
\psfig{figure=images/eps/ex3-4.eps}
\label{s24}
\caption{Box with a widget label}
\end{figure}
