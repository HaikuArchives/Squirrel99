
\section{The widget BarberPole}

A {\it BarberPole} is a widget that display a barberpole that can be started to stoped. Usually this widget is used to show activity without knowing how long it's gonna take.

\subsection{Construction}

The primitive {\tt BarberPole} is used to build a new widget. Its syntax is :\\

{\tt BarberPole} {\it list (word)}\\

The first input is a list of two number that indicate the size of the widget in pixel (with height). If a second input is given, it must be the word {\tt "left} or {\tt "right}. It indicate the direction the BarberPole must run. By default it is left to right.

\subsection{Methods}

A BarberPole widget has two methods :\\

{\bf start\index{BarberPole@\textbf{BarberPole}!Methods!start}} 
\begin{verbatimtab}
$barber~start
\end{verbatimtab}
The method start the widget spinning.\\

{\bf stop\index{BarberPole@\textbf{BarberPole}!Methods!stop}} 
\begin{verbatimtab}
$barber~stop
\end{verbatimtab}
The method stop the widget from spinning.\\

\subsection{Configuration}

This widget have nothing particular. To control the color of the barberpole, use {\tt high.color} and {\tt low.color}.

\subsection{Hooks}

This widget doesn't have any particular hook.

\subsection{Example}

\example{ex47}	
\begin{listing}{1}
make "MyWin Window "titled 'BarberPole' [100 100] "not.closable
make "pole BarberPole [30 10]
$pole~config "high.color "set :Blue
Glue :MyWin "top [] :pole
$MyWin~show

$pole~start
for ["i 1 2] {
	wait 1	
}
$pole~stop

$MyWin~quit
\end{listing}

In this example, we create a simple Window with only a BarberPole in it, then we set it spinning and we execute a loop that will take 2 seconds to complete, then we stop the the BarberPole and we ask the Window to quit. :

\begin{figure}[ht!]
\centering
\psfig{figure=images/eps/ex3-49.eps,width=41pt,height=40pt}
\label{s47}
\caption{Spining BarberPole}
\end{figure}
