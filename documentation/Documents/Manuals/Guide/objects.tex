\chapter{Standard Objects}

This chapter describes the various {\em Objects} than \squirrel handle.

\section{Application Messaging}
 
\cbstart 
Using the object {\tt Message}, it is possible with \squirrel to send or receive messages from/to another application or to receive a drag and drop notication on a widget.\\
\cbend

The primitive {\tt Message} is used to create a message object. The primitive accept as only input a number (integer or exadecimal) which is a value that captures what the message is about.\\

A message could store pretty mutch any basics \squirrel object like : word, number or list. When a list is given, the list elements will be added with the same data name. They must be of the exact same object type, else the method {\tt add} will fail.\\

A \squirrel script accept to receive message by using a hook function that the script must define : {\tt \verb+_+MessageReceived}. This function has only one input which is the received message:

\example{ex101}
\begin{verbatimtab}
to _MessageReceived :msg
	
	switch ($msg~what) {
		case 0x80 {
			print 'Received :' $msg~find "test
			$msg~reply Message 0xFF			
		}	
	}	
	
end
\end{verbatimtab}

When adding data in a message, it is possible to specify the exact datatype. The following table list them all:

\begin{table}[ht!]
\begin{center}
\begin{tabular}{|c|l|}
\hline
{\tt "int8} & an int8 or uint8 \\
\hline
{\tt "int16} & an int16 or uint16 \\
\hline
{\tt "int32} & an int32 or uint32 \\
\hline
{\tt "int64} & an int64 or uint64 \\
\hline
{\tt "float} & a float \\
\hline
{\tt "double} & a double \\
\hline
{\tt "string} & a string \\
\hline
{\tt "point} & a BPoint \\
\hline
{\tt "rect} & a BRect \\
\hline
\end{tabular}
\end{center}
\caption{Message datatype}
\label{messtype}
\end{table}

A message object has severals methods that allow to get/set the datas stored into it, as well as sending it to an application:

\subsubsection*{Message\textasciitilde add} \index{\verb+Message+!\verb+~add+}

{\tt \verb+~+add (word | string) thing (word)} 
\newline\newline
Put data in the message. The first input is the name of the data to add, the second is the value to store. If a third input is given it must be a word that describe the exact datatype to use. The method will output {\tt true} if no error occurs, {\tt false} else.  
\begin{verbatimtab}
@> $msg~add "length 56.78
\end{verbatimtab}

\subsubsection*{Message\textasciitilde delivered} \index{\verb+Message+!\verb+~delivered+}

{\tt \verb+~+delivered} 
\newline\newline
Output {\tt true} if the message has been delivered and so that it is possible to send a reply to that message. 
\begin{verbatimtab}
@> print $msg~delivered
true
\end{verbatimtab}

\subsubsection*{Message\textasciitilde empty} \index{\verb+Message+!\verb+~empty+}
{\tt \verb+~+empty} 
\newline\newline
Remove all data from a message. Output {\tt true} if no error is encounter. 
\begin{verbatimtab}
@> $msg~add "test 'hello there'
@> $msg~empty
@> print $msg~has "test
false
\end{verbatimtab}

\subsubsection*{Message\textasciitilde find} \index{\verb+Message+!\verb+~find+}

{\tt \verb+~+find word | string} 
\newline\newline
Output a new object created from the data(s) read in the message for the name given as input of the method. An error is raised if the name is not found.
\begin{verbatimtab}
@> $msg~add "test 'hello there'
@> print $msg~find "test
hello there
\end{verbatimtab}

\subsubsection*{Message\textasciitilde has} \index{\verb+Message+!\verb+~has+}

{\tt \verb+~+has (word | string) (word)} 
\newline\newline
Output {\tt true} if a data is found for the name given as first input. If a second input is given, it should match the datatype of the data found. 
\begin{verbatimtab}
@> $msg~add "test 'hello there'
@> print $msg~has "test
true
@> print $msg~has "test "int32
false
\end{verbatimtab}

\subsubsection*{Message\textasciitilde is.empty} \index{\verb+Message+!\verb+~is.empty+}

{\tt \verb+~+is.empty} 
\newline\newline
Output {\tt true} if a the message is empty of any data.
\begin{verbatimtab}
@> $msg~add "test 'hello there'
@> print $msg~is.empty
false
\end{verbatimtab}
 
\subsubsection*{Message\textasciitilde is.remote} \index{\verb+Message+!\verb+~is.remote+}

{\tt \verb+~+is.remote} 
\newline\newline
Output {\tt true} if a the message has been sended by another application.
\begin{verbatimtab}
@> print $msg~is.remote
true
\end{verbatimtab} 
 
\subsubsection*{Message\textasciitilde is.reply} \index{\verb+Message+!\verb+~is.reply+}

{\tt \verb+~+is.reply} 
\newline\newline
Output {\tt true} if a the message is a reply to another message.
\begin{verbatimtab}
@> $msg~add "test 'hello there'
@> print $msg~is.reply
false
\end{verbatimtab} 
 
\subsubsection*{Message\textasciitilde is.waiting} \index{\verb+Message+!\verb+~is.waiting+}

{\tt \verb+~+is.waiting} 
\newline\newline
Output {\tt true} if a the sender of the message is waiting for a reply.
\begin{verbatimtab}
@> print $msg~is.waiting
false
\end{verbatimtab} 
 
\subsubsection*{Message\textasciitilde names} \index{\verb+Message+!\verb+~names+}

{\tt \verb+~+names} 
\newline\newline
Output a list of all the data names in the message.
\begin{verbatimtab}
@> $msg~add "num 23.56
@> $msg~add "str 'what can I do ?'
@> $msg~add "foo [34 65 23 90]
@> show $msg~names
['num' 'str' 'foo']
\end{verbatimtab} 

\subsubsection*{Message\textasciitilde previous} \index{\verb+Message+!\verb+~previous+}

{\tt \verb+~+previous} 
\newline\newline
Output the message to which the current message is a reply. {\tt false} is outputted if the message is not a reply.
\begin{verbatimtab}
@> make "org $rep~previous
\end{verbatimtab} 

\subsubsection*{Message\textasciitilde rem} \index{\verb+Message+!\verb+~rem+}

{\tt \verb+~+rem word | string} 
\newline\newline
Remove a data from the message. Output {\tt true} if it succes.
\begin{verbatimtab}
@> $msg~rem "test
\end{verbatimtab} 

\subsubsection*{Message\textasciitilde replace} \index{\verb+Message+!\verb+~replace+}

{\tt \verb+~+replace (word | string) thing (word)} 
\newline\newline
Replace data in the message. The first input is the name of the data to be replaced, the second if the value to store. If a third input is given is must be a word that describe the exact datatype to use. The method will output {\tt true} if no error occurs, {\tt false} else. The datatype MUST be the same than the data to replace.  
\begin{verbatimtab}
@> $msg~add "length 56.78
@> $msg~replace "length 34
\end{verbatimtab}

\subsubsection*{Message\textasciitilde reply} \index{\verb+Message+!\verb+~reply+}

{\tt \verb+~+reply message (word)} 
\newline\newline
Send a message as the reply to the message given as first input. The message is NOT destroyed by this method. A message could then be send several time. If a second input is given, it must be the word {\tt "reply}. In this case, the method will block until it receives a reply from the target of the message. The method outputs {\tt true} if the reply has been send.
\begin{verbatimtab}
make "r Message 0x78
$r~add "data 45.8
$msg~reply :r 
\end{verbatimtab} 
 
\subsubsection*{Message\textasciitilde send} \index{\verb+Message+!\verb+~send+}

{\tt \verb+~+send string (word)} 
\newline\newline
Send the message to an application whose signature is given as first input. The message is NOT destroyed by this method. A message could then be send several time. If a second input is given, it must be the word {\tt "reply}. In this case, the method will block until it receives a reply from the target of the message. The method output {\tt true} if the message has been send.
\begin{verbatimtab}
@> $msg~send 'application/x-vnd.Foo-Inc.MyApp'
\end{verbatimtab}  
 
\subsubsection*{Message\textasciitilde timeout} \index{\verb+Message+!\verb+~timeout+}

{\tt \verb+~+timeout word (number)} 
\newline\newline
Set the timeout of the message when sending it, if the first input is {\tt "send} or output the current timout if no second input is given. If the first inputs is {\tt "reply} the timout will be ON to wait for a reply. A value of {\tt -1} sets an infinite timeout.  
\begin{verbatimtab}
@> $msg~timeout "reply 50000
\end{verbatimtab} 

\subsubsection*{Message\textasciitilde what} \index{\verb+Message+!\verb+~what+}

{\tt \verb+~+what} 
\newline\newline
Output the number that defines what the message is all about. 
\begin{verbatimtab}
@> print $msg~what
128
\end{verbatimtab} 

\cbstart
\section{Image object}

\squirrel 5.3 introduces a new kind of object that allow to handle images. This object can load and manipulate any kind of images for which there is an installed translator.\\

The primitive {\tt Image} is used to create an {\it Image} object. It accept either a filename or a given width and height (in a list) in pixels. When no filename is given, the image will be created empty.\\

The primitive {\tt trans.mime}\index{\verb+trans.mime+} outputs a list of all MIME type for which there is a Translator, whiles the primitive {\tt trans.name}\index{\verb+trans.name+} outputs a list of the image format names.\\  

An {\it Image} object had severals methods described bellow:

\subsubsection*{Image\textasciitilde height} \index{\verb+Image+!\verb+~height+}

{\tt \verb+~+height}  
\newline\newline
Output the height in pixel of the image. 
\begin{verbatimtab}
@> $img~load 'ariane5.jpg'
@> print $img~height
600
\end{verbatimtab}

\subsubsection*{Image\textasciitilde length} \index{\verb+Image+!\verb+~length+}

{\tt \verb+~+length} 
\newline\newline
Output the memory used by the Image (in bytes).
\begin{verbatimtab}
@> $img~length
1228800
\end{verbatimtab}

\subsubsection*{Image\textasciitilde load} \index{\verb+Image+!\verb+~load+}

{\tt \verb+~+load string}  
\newline\newline
Load an file into the image and discard any image already loaded. The method will select the correct Translator according to the file's MIME type.
\begin{verbatimtab}
@> $img~load 'myimage.gif'
\end{verbatimtab}

\subsubsection*{Image\textasciitilde mime} \index{\verb+Image+!\verb+~mime+}

{\tt \verb+~+mime word (string)}  
\newline\newline
Set or get the MIME type of the image. If the MIME type is set it will be used when the image is save.
\begin{verbatimtab}
@> $img~load 'myimage.jpg'
@> print $img~mime "get
image/jpeg 
\end{verbatimtab}

\subsubsection*{Image\textasciitilde path} \index{\verb+Image+!\verb+~path+}

{\tt \verb+~+path word (string)}  
\newline\newline
Set or get the path of the image.
\begin{verbatimtab}
@> $img~load 'myimage.jpg'
@> $img~path "set '/this/place/this_name.gif'
@> $img~save
\end{verbatimtab}

\subsubsection*{Image\textasciitilde save} \index{\verb+Image+!\verb+~save+}

{\tt \verb+~+save (string word)}  
\newline\newline
Save the image into a file. If no input is given, the method use the path and MIME type of the image object. Otherwise, the input must be the path of the file to create and the name of the format to store the image in.
\begin{verbatimtab}
@> $img~load 'myimage.jpg'
@> $img~save 'myimage.gif' "GIF
\end{verbatimtab}

\subsubsection*{Image\textasciitilde valid?} \index{\verb+Image+!\verb+~valid?+}

{\tt \verb+~+valid?}  
\newline\newline
Output {\tt true} if the image is a correct image. When an image object is created from a file, the image is only valid if the loading has been a sucess. 
\begin{verbatimtab}
@> $img~load 'myimage.cpp'
@> print $img~valid?
false
\end{verbatimtab}

\subsubsection*{Image\textasciitilde width} \index{\verb+Image+!\verb+~width+}

{\tt \verb+~+width}  
\newline\newline
Output the width in pixel of the image. 
\begin{verbatimtab}
@> $img~load 'ariane5.jpg'
@> print $img~width
800
\end{verbatimtab}
\cbend
