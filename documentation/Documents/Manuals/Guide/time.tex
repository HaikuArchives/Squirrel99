\section{Time} 

This Add-on contains primitives to access time.

\subsection*{clock} \index{\verb+clock+} 
 
{\tt clock} 
\newline\newline
Output an approximation for the current processor time used by the program.
\begin{verbatimtab} 
@> print clock
1
\end{verbatimtab}

\subsection*{ctime} \index{\verb+ctime+} 
 
{\tt ctime} {\it long}
\newline\newline
Output the time in a string format from a time value given as the second input.
\begin{verbatimtab} 
@> print ctime time
Thu Aug 19 14:30:06 1999
\end{verbatimtab}

\subsection*{c2sec} \index{\verb+c2sec+} 
 
{\tt c2sec} {\it long}
\newline\newline
Output the time in seconds from a clock value.
\begin{verbatimtab} 
@> make "t0 clock
@> do_something_rather_long
@> print 'Time elapsed: ' c2sec :t0-(clock)
0.34
\end{verbatimtab}

\subsection*{time} \index{\verb+time+} 
 
{\tt time} 
\newline\newline
Output the time elapsed in seconds since Epoch (00:00:00 UTC, January 1, 1970).
\begin{verbatimtab} 
@> print time
935108764
\end{verbatimtab}

\subsection*{timing} \index{\verb+timing+} 
 
{\tt timing block | (word things ...)} 
\newline\newline
Output the elapsed time in microseconds of the execution of the block or function.
\begin{verbatimtab} 
@> print timing "foo 45 "hello
4234
\end{verbatimtab}
