\section{Control} 
 
Most control structures are implemented deep inside the interpreter, but some of the simplest are in this Add-on. 
 
\subsection*{call} \index{\verb+call+} 
 
{\tt call} {\it word thing1 thing2 thing3 ...} 
\newline\newline 
The command {\tt call} executes commands (primitive or procedure) given as its first input. All the other inputs will be treated as regular inputs to this command. 
\begin{verbatimtab} 
@> call "print 'This is a test:' sum 4 5
This is a test: 9
\end{verbatimtab}
 
\subsection*{break} \index{\verb+break+} 
 
{\tt break} 
\newline\newline 
This order only makes sense within a loop. It will stop the execution of the loop and return without executing the remaining orders in the loop block. 
\begin{verbatimtab} 
for ["i 1 100] { 
 
	if (myfunc :i (random 45))<34 {break} 
 
	dosomething_with :i 
}
\end{verbatimtab}
 
\subsection*{continue} \index{\verb+continue+} 
 
{\tt continue} 
\newline\newline 
This order only makes sense within a loop. It will advance in the execution of the loop and not execute the remaining orders in the loop block. 
\begin{verbatimtab} 
for ["i 1 100] { 
 
	if :i%10 {continue} 
 
	print :i 
} 
\end{verbatimtab}
 
Will produce: 
 
\begin{verbatimtab} 
10 
20 
30 
40 
50 
60 
70 
80 
90 
100 
\end{verbatimtab}
  
\subsection*{stop} \index{\verb+stop+} 
 
{\tt stop} 
\newline\newline 
Stop the execution of a function. Execute the next order after calling the function. 
\begin{verbatimtab} 
to foo :a 
	if :a=0 {stop} 
 	print :a 
end 
\end{verbatimtab}
 
\subsection*{test} \index{\verb+test+} 
 
{\tt test} {\it expression} 
\newline\newline 
Evaluate the expression in input and remember the result. The result will be used by an {\tt iftrue} and {\tt iffalse}. 
\begin{verbatimtab} 
@> test :a < 10 
\end{verbatimtab}
  
\subsection*{wait} \index{\verb+wait+} 
 
{\tt wait} {\it number} 
\newline\newline 
Suspend the execution for a {\it number} seconds. The input must be an integer. 
\begin{verbatimtab} 
@> wait 2 
\end{verbatimtab}
