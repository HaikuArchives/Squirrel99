\section{Threading}

\squirrel supports {\em multi-threading} with this Add-on. A thread could be either a function or a block and will run concurrently and asynchronously to any other \squirrel thread. A thread has an ID which is unique. Using this ID as its reference, it's possible to kill, suspend, or wait for the end of thread. Each thread has a priority. Higher priority threads will run more often and will terminate faster than a lower priority thread. The following table contains all the priorities supported by \squirrel:

\begin{table}[h!]
\centering
\begin{tabular}{|c|}
\hline
\bf Name \\
\hline
\tt "low \\
\hline
\tt "normal \\
\hline
\tt "display \\
\hline
\tt "urgent\_display \\
\hline
\tt "realtime\_display \\
\hline
\tt "urgent \\
\hline
\tt "realtime \\
\hline
\end{tabular}
\caption{Thread priority (from low to high)}
\end{table}

When the thread is a function, the function can output a value. In case another thread is waiting, the output of the function will be returned to the {\tt Thread.waitfor} primitive.

It may be necessary in a threading script to use a locker (or several lockers) to insure that a critical section of the script will be protected from concurrent threads. For example, the modification of the value of a global variable shared between several threads.

\subsection*{Locker} \index{\verb+Locker+}

{\tt Locker}
\newline\newline
Output a new locker object. This object has a two member functions that locks it  {\tt \verb+~+lock} or unlocks it  {\tt \verb+~+unlock}.
\begin{verbatimtab}
make "mylock Locker
make "sum 0

to thread_1
	mylock~lock
	incr "sum 10
	mylock~unlock
end

make "th1 Thread "normal "thread_1
make "th2 Thread "normal "thread_1
Thread.hop :th1 :th2
\end{verbatimtab}

\subsection*{Priority} \index{\verb+Priority+}

{\tt Priority} {\it (word)}
\newline\newline
Output or set the priority of the calling thread. If an input is given, it should be a valid word.
\begin{verbatimtab}
@> Priority "low
\end{verbatimtab}

\subsection*{snooze} \index{\verb+snooze+}

{\tt snooze} {\it integer}
\newline\newline
Pause the calling thread for a given number of microseconds.
\begin{verbatimtab}
@> snooze 10000
\end{verbatimtab}

\subsection*{Thread} \index{\verb+Thread+}

{\tt Thread} {\it word block \verb+|+ (word (thing)*)}
\newline\newline
Create a new thread. The first input is the priority of the thread. The second input could either be a block or a word. When the name of a function is given, the restart inputs of the primitives will be fed to the function when it is executed. The primitive outputs the {\em thread id} of the thread. The thread does not start at the end of this primitive.
\begin{verbatimtab}
to thread_1 :iter
	make.local "s 0
	for ["i 1 :iter] {
		make "s :i		
	}
	output :s
end

@> make "th1 Thread "low "thread_1 20
\end{verbatimtab}

\subsection*{ThreadID} \index{\verb+ThreadID+}

{\tt ThreadID}
\newline\newline
Output the {\em thread id} of the calling thread.
\begin{verbatimtab}
@> print ThreadID
367
\end{verbatimtab}

\subsection*{Thread.hop} \index{\verb+Thread.hop+}

{\tt Thread.hop} {\it (number)+}
\newline\newline
Start or resume one or more thread(s). The input must be the {\em thread ids} of the threads to run.
\begin{verbatimtab}
@> Thread.hop :th1
\end{verbatimtab}

\subsection*{Thread.hoping} \index{\verb+Thread.hoping+}

{\tt Thread.hoping}
\newline\newline
Output a list of the {\em thread ids} of all the running threads.
\begin{verbatimtab}
@> show Thread.hoping
[453]
\end{verbatimtab}

\subsection*{Thread.kill} \index{\verb+Thread.kill+}

{\tt Thread.kill} {\it (number)+}
\newline\newline
Kill one or more thread(s). The input must be the {\em thread ids} of the threads to kill.
\begin{verbatimtab}
@> Thread.kill :th1
\end{verbatimtab}

\subsection*{Thread.priority} \index{\verb+Thread.priority+}

{\tt Thread.priority} {\it number (word)}
\newline\newline
Output or set the priority of a thread given as the first input. If a second input is given, it should be a valid word.
\begin{verbatimtab}
@> print Thread.priority :th1
normal
\end{verbatimtab}

\subsection*{Thread.suspend} \index{\verb+Thread.suspend+}

{\tt Thread.suspend} {\it (number)+}
\newline\newline
Suspend one or more thread(s). The input must be the {\em thread ids} of the threads to suspend.
\begin{verbatimtab}
@> Thread.suspend :th1
\end{verbatimtab}

\subsection*{Thread.waitfor} \index{\verb+Thread.waitfor+}

{\tt Thread.waitfor} {\it number \verb+|+ (word (number)+)}
\newline\newline
Wait for the end of one or more thread(s). When the primitive should wait for several threads, the first input must be the word : {\tt "all} or {\tt "first}. When {\tt "all} is used, the primitive will wait for the end of all the threads before producing a list of the thread's output values. When {\tt "first} is used, the primitive will return when the first thread ends and then outputs the thread's output values.
\begin{verbatimtab}
@> print Thread.waitfor :th1
\end{verbatimtab}

\subsection*{Wait} \index{\verb+Wait+}

{\tt Wait} {\it word}
\newline\newline
The calling thread waits for the variable given as input to be updated by another thread. The primitive output {\tt true} if the thread as wait, {\tt false} else.
\begin{verbatimtab}
@> Wait "mtvar
\end{verbatimtab}
