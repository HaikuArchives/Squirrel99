\section{Mathematics} 
 
To perform mathematical operations such as {\tt max}, and {\tt rsin}, you may use this Add-on. It provides random number generations and contains a large amount of Mathematical primitives. 
 
 
\subsection*{abs} \index{\verb+abs+} 
 
{\tt abs} {\it number} 
\newline\newline 
Output the absolute value of the input 
\begin{verbatimtab} 
@> print abs -4.78 
4.78 
\end{verbatimtab} 
 
 
\subsection*{av} \index{\verb+av+} 
 
{\tt av} {\it thing1 thing2 thing3 thing4 ...} 
\newline\newline 
Output the average value of the input. The input may be a {\it number} or a {\it list}. Any input of other types will be ignored. 
\begin{verbatimtab} 
@> print av 3 6 2 9.6 [3 2 4 0.5] 
3.762 
\end{verbatimtab} 
 
\subsection*{ceil} \index{\verb+ceil+} 
 
{\tt ceil} {\it number} 
\newline\newline 
Output the smallest integer that is not less than the input. 
\begin{verbatimtab} 
@> print ceil 4.576
5 
\end{verbatimtab} 

\subsection*{deg2rad} \index{\verb+deg2rad+} 
 
{\tt deg2rad} {\it number} 
\newline\newline 
Output the equivalent angle in radians for the degree given (converting degrees to radians)
\begin{verbatimtab} 
@> print deg2rad 120 
2.094
\end{verbatimtab}
 
\subsection*{difference} \index{\verb+difference+} 
 
{\tt difference} {\it number1 number2} 
\newline\newline 
Output the difference between {\it number1} and {\it number2}.
\begin{verbatimtab} 
@> print difference 4 5 
1 
\end{verbatimtab}

\subsection*{dacos} \index{\verb+dacos+} 
 
{\tt dacos} {\it number} 
\newline\newline 
Output the arccosine of the input given in degrees 
\begin{verbatimtab} 
@> print dacos 45 
0.9033
\end{verbatimtab}

\subsection*{dasin} \index{\verb+dasin+} 
 
{\tt dasin} {\it number} 
\newline\newline 
Output the arcsine of the input given in degrees 
\begin{verbatimtab} 
@> print dasin 180
nan
\end{verbatimtab}

\subsection*{datan} \index{\verb+datan+} 
 
{\tt datan} {\it number} 
\newline\newline 
Output the arctangent of the input given in degrees
\begin{verbatimtab} 
@> print datan 45
0.6658
\end{verbatimtab}

\subsection*{dcos} \index{\verb+dcos+} 
 
{\tt dcos} {\it number} 
\newline\newline 
Output the cosine of the input given in degrees 
\begin{verbatimtab} 
@> print dcos 180 
-1
\end{verbatimtab} 

\subsection*{dcosh} \index{\verb+dcosh+} 
 
{\tt rcosh} {\it number} 
\newline\newline 
Output the hyperbolic cosine of the input given in degrees 
\begin{verbatimtab} 
@> print dcosh 180
11.59
\end{verbatimtab}

\subsection*{dsin} \index{\verb+dsin+} 
 
{\tt dsin} {\it number} 
\newline\newline 
Output the sine of the input given in degrees 
\begin{verbatimtab} 
@> print dsin 60 
0.866
\end{verbatimtab}

\subsection*{dsinh} \index{\verb+dsinh+} 
 
{\tt dsinh} {\it number} 
\newline\newline 
Output the hyperbolic sine of the input given in degrees 
\begin{verbatimtab} 
@> print dsinh 180
11.55
\end{verbatimtab}

\subsection*{dtan} \index{\verb+dtan+} 
 
{\tt dtan} {\it number} 
\newline\newline 
Output the tangent of the input given in degrees 
\begin{verbatimtab} 
@> print dtan 140 
-0.8391
\end{verbatimtab}

\subsection*{dtanh} \index{\verb+dtanh+} 
 
{\tt dtanh} {\it number} 
\newline\newline 
Output the hyperbolic-tangent of the input given in degrees 
\begin{verbatimtab} 
@> print dtanh 78 
0.8767
\end{verbatimtab}

\subsection*{erf} \index{\verb+erf+} 
 
{\tt erf} {\it number} 
\newline\newline 
Output the error function of the input. 
\begin{verbatimtab} 
@> print erf 4.5 
1 
\end{verbatimtab}
 
\subsection*{erfc} \index{\verb+erfc+} 
 
{\tt erfc} {\it number} 
\newline\newline 
Output the complementary error function of the input. 
\begin{verbatimtab} 
@> print erfc 4.5 
1.96616e-10 
\end{verbatimtab}
 
\subsection*{exp} \index{\verb+exp+} 
 
{\tt exp} {\it number} 
\newline\newline 
Output the exponential value of the input. 
\begin{verbatimtab} 
@> print exp 4.7 
109.95 
\end{verbatimtab} 
 
\subsection*{floor} \index{\verb+floor+} 
 
{\tt floor} {\it number} 
\newline\newline 
Output the largest integer that is not greater than the input. 
\begin{verbatimtab} 
@> print floor 4.576 
4 
\end{verbatimtab} 
 
\subsection*{gamma} \index{\verb+gamma+} 
 
{\tt gamma} {\it number} 
\newline\newline 
Output the log gamma function of the input. 
\begin{verbatimtab} 
@> print gamma 4.5 
2.45374 
\end{verbatimtab} 
 
\subsection*{hypot} \index{\verb+hypot+} 
 
{\tt hypot} {\it number1 number2} 
\newline\newline 
Output the Euclidean distance function of the input 
\begin{verbatimtab} 
@> print hypot 2.5 6.8 
7.245 
\end{verbatimtab} 
 
\subsection*{incr} \index{\verb+incr+} 
 
{\tt incr} {\it word (number)} 
\newline\newline 
Increment the value stored in a variable by 1 or by the number specified as the second input. The first input must be a known variable name. 
\begin{verbatimtab} 
@> make "a 1 
@> incr "a -1 
@> print :a 
0 
\end{verbatimtab}
 
\subsection*{int} \index{\verb+int+} 
 
{\tt int} {\it number} 
\newline\newline 
Output a cast of the input into an integer. 
\begin{verbatimtab} 
@> print int 4.576 
4 
\end{verbatimtab}
 
\subsection*{log10} \index{\verb+log10+} 
 
{\tt log10} {\it number} 
\newline\newline 
Output the logarithm (base 10) of the input. 
\begin{verbatimtab} 
@> print log10 5000 
3.699 
\end{verbatimtab}
 
\subsection*{ln} \index{\verb+ln+} 
 
{\tt ln} {\it number} 
\newline\newline 
Output the natural logarithm of the input. 
\begin{verbatimtab} 
@> print ln 5000 
8.517 
\end{verbatimtab}
 
 
\subsection*{max} \index{\verb+max+} 
 
{\tt max} {\it thing1 thing2 thing3 thing4 ...} 
\newline\newline 
Output the maximum value of the inputs.  The inputs may be {\it numbers} or {\it lists}. Any input of other types 
will be ignored. 
\begin{verbatimtab} 
@> print max 3 6 2 9.6 [3 2 4 0.5] 
9.6 
\end{verbatimtab} 
 
\subsection*{min} \index{\verb+min+} 
 
{\tt min} {\it thing1 thing2 thing3 thing4 ...} 
\newline\newline 
Output the minimum value of the inputs.  The inputs may be {\it numbers} or {\it lists}. Any input of other types 
will be ignored. 
\begin{verbatimtab} 
@> print min 3 6 2 9.6 [3 2 4 0.5] 
0.5 
\end{verbatimtab}
 
\subsection*{minus} \index{\verb+minus+} 
 
{\tt minus} {\it number} 
\newline\newline 
Output the negative value of the input.  
\begin{verbatimtab} 
@> print minus -5 
5 
@> print minus 45.78 
-45.78 
\end{verbatimtab}
 
\subsection*{product} \index{\verb+product+} 
 
{\tt product} {\it thing1 thing2 thing3 thing4 ...} 
\newline\newline 
Output the product of all the inputs.  The inputs may be {\it numbers} or {\it lists}. Any input of other types 
will be ignored. 
\begin{verbatimtab} 
@> print product 3 6 2 9.6 [3 2 4 0.5] 
4147.2 
\end{verbatimtab}
 
\subsection*{quotient} \index{\verb+quotient+} 
 
{\tt quotient} {\it number1 number2 } 
\newline\newline 
Output the quotient of {\it number1} over {\it number2.} 
\begin{verbatimtab} 
@> print quotient 5 7 
0.714 
\end{verbatimtab}
 
\subsection*{modulo} \index{\verb+modulo+} 
 
{\tt modulo} {\it number1 number2 } 
\newline\newline 
Output the remainder from performing the integer quotient of {\it number1} over {\it number2}. Both numbers must be integers. 
\begin{verbatimtab} 
@> print modulo 10 2 
0 
\end{verbatimtab}
 
\subsection*{power} \index{\verb+power+} 
 
{\tt power} {\it number1 number2} 
\newline\newline 
Output the power of {\it number1} by {\it number2}
\begin{verbatimtab} 
@> print power 3.4 5 
454.354
\end{verbatimtab} 
 
\subsection*{racos} \index{\verb+racos+} 
 
{\tt racos} {\it number} 
\newline\newline 
Output the arccosine of the input given in radians 
\begin{verbatimtab} 
@> print racos 0.4 
1.159 
\end{verbatimtab}

\subsection*{rad2deg} \index{\verb+rad2deg+} 
 
{\tt rad2deg} {\it number} 
\newline\newline 
Output the corresponding angle in degrees, given the angle in radians  (converting from radians into degrees)
\begin{verbatimtab} 
@> print rad2deg 2.094
120
\end{verbatimtab}
 
\subsection*{random} \index{\verb+random+} 
 
{\tt random} {\it (min) max} 
\newline\newline 
Output a random number between the given min and max.  If only one input is given, it will be interpreted as the max and the output will be a number between 0 and max. 
\begin{verbatimtab} 
@> print random 40 
15 
\end{verbatimtab}
 
\subsection*{rcos} \index{\verb+rcos+} 
 
{\tt rcos} {\it number} 
\newline\newline 
Output the cosine of the input given in radians 
\begin{verbatimtab} 
@> print rcos 60 
-0.952 
\end{verbatimtab}
 
\subsection*{rcosh} \index{\verb+rcosh+} 
 
{\tt rcosh} {\it number} 
\newline\newline 
Output the hyperbolic cosine of the input given in radians 
\begin{verbatimtab} 
@> print rcosh 0.56 
1.160 
\end{verbatimtab}
 
\subsection*{rasin} \index{\verb+rasin+} 
 
{\tt rasin} {\it number} 
\newline\newline
Output the arcsine of the input given in radians 
\begin{verbatimtab} 
@> print rasin 0.4 
0.411 
\end{verbatimtab}
 
\subsection*{rsin} \index{\verb+rsin+} 
 
{\tt rsin} {\it number} 
\newline\newline 
Output the sine of the input given in radians 
\begin{verbatimtab} 
@> print rsin 60 
-0.304 
\end{verbatimtab}
 
\subsection*{rsinh} \index{\verb+rsinh+} 
 
{\tt rsinh} {\it number} 
\newline\newline 
Output the hyperbolic sine of the input given in radians 
\begin{verbatimtab} 
@> print rsinh 0.56 
0.589 
\end{verbatimtab}
 
\subsection*{ratan} \index{\verb+ratan+} 
 
{\tt ratan} {\it number} 
\newline\newline 
Output the arctangent of the input given in radians 
\begin{verbatimtab} 
@> print ratan 0.6 
0.540 
\end{verbatimtab}
 
\subsection*{rtan} \index{\verb+rtan+} 
 
{\tt rtan} {\it number} 
\newline\newline 
Output the tangent of the input given in radians 
\begin{verbatimtab} 
@> print rtan 60 
0.32 
\end{verbatimtab}
 
\subsection*{rtanh} \index{\verb+rtanh+} 
 
{\tt rtanh} {\it number} 
\newline\newline 
Output the hyperbolic tangent of the input given in radians 
\begin{verbatimtab} 
@> print rtanh 0.7 
0.604 
\end{verbatimtab}
 
\subsection*{sqrt} \index{\verb+sqrt+} 
 
{\tt sqrt} {\it number} 
\newline\newline 
Output the square root of the input. 
\begin{verbatimtab} 
@> print sqrt 25 
5 
\end{verbatimtab} 
 
\subsection*{sum} \index{\verb+sum+} 
 
{\tt sum} {\it thing1 thing2 thing3 thing4 ...} 
\newline\newline 
Output the sum of the all the inputs.  The inputs may be {\it numbers} or {\it lists}. Any input of other types 
will be ignored. 
\begin{verbatimtab} 
@> print sum 3 6 2 9.6 [3 2 4 0.5] 
30.01 
\end{verbatimtab}
