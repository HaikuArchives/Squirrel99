\chapter*{About this document\markboth{About this document}{}}
\addcontentsline{toc}{chapter}{About this Developer's Guide}

\begin{quote}

\pagestyle{fancy}
\pagenumbering{arabic}

Squirrel is a programming language in the Logo family. There are some distinct differences between \squirrel and Logo, some stemming from \squirrel taking advantage of advanced features of the Be operating system (\beos).\\

At this time, neither \squirrel nor this document is perfect. We would appreciate notification of any errors found.\\

This guide is divided into four parts:

\begin{description}
\item [Getting started] introduces \squirrel and describes how to install it
\item [Squirrel basics] shows, mostly by examples, how to use the \squirrel language
\item [Primitives] lists and describes all the primitives
\item [Release notes] contains pertinent information on the releases
\end{description}

It should be understood that several additional features will be included in upcoming releases. These features include advanced scripting capabilities and other additions for the Add-ons. Some Add-ons will be complete while others will not. An Add-on API in the near future will allow third party additions, which will enhance \squirrel 's usefulness.\\

We have included several documentation conventions in this document. These are: \\

\begin{itemize}
\item All code elements are presented in a distinct font like {\tt print "foo}
\item Primitive syntax is often a combination of code element and italic font. The part in italic is always the input to the primitive.
\item Primitive inputs use special kind of symbols :
\begin{itemize}
\item {\it \bf (word)} indicate that the input is optional
\item {\it \bf word \verb?|? number} indicate that the input can be either a word or a number
\item {\it \bf (word)+} indicate that the primitive can take on several words as input, but at least one is required.
\item {\it \bf (word)*} indicate that the primitive can take on several words as input, but one is optional.
\end{itemize}
\end{itemize}

\cbstart
A bar in the right or left margin helps to locate an update or addition (from the previous revision) in this document.\\ 
\cbend

A big {\em Mahalo\footnote{"Thank you" in Hawaiian}} to (in chronological order) :

\begin{itemize}
\item Henry van Eyken
\item Susan Banh\footnote{and all my love}
\item Ulrich "scholly" Scholz
\item Guido Soranzio
\item Jonas Sundstr\"om
\end{itemize}

for their much appreciated contributions towards rewriting and editing this document !\\

Please enjoy reading this manual and have fun with \squirrel !\\

Jean-Louis, \today

